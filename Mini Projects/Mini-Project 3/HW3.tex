\documentclass[11pt]{article}
\documentclass[11pt]{article}
\documentclass{article}
\usepackage{libertine}
\usepackage{amssymb}
\usepackage{amsmath}
\usepackage[makeroom]{cancel}

% To produce a letter size output. Otherwise will be A4 size.
\usepackage[letterpaper]{geometry}
% To produce a letter size output. Otherwise will be A4 size.
\usepackage[letterpaper]{geometry}

% For enumerated lists using letters: a. b. etc.
\usepackage{enumitem}

\topmargin -.5in
\textheight 9in
\oddsidemargin -.25in
\evensidemargin -.25in
\textwidth 7in
\usepackage{pifont}

\begin{document}





\begin{center}
    {\Large Valentinno Cruz\\
    Homework Assignment 3\\
    March. 9th, 2021\\}

\end{center}

%-----------        ----------- 
%----------- Part 1 ----------- 
%-----------        ----------- 


\begin{flushleft}
{\Large 1. Mathematical Proofs}
\end{flushleft}



%========= Problem 1 .===========
\begin{enumerate}

\begin{flushleft}
{\large 1. The square of an even number is even.}
\end{flushleft}

\begin{itemize}

\item \textbf{Solution}\\
\large 1. Even numbers = $2n$\\
\large 2. Square of an even number = $(2n)(2n)$\\
\large 3. = $4n^2$\\
\large 4. = $2(2n^2)$\\
\large 5. $2(2n^2)$ can be divided by 2 since it is a multiple of 2\\
\large 6. \therefore \hspace{.2cm} $Even$\\

\end {itemize}
\end {enumerate}


%========= Problem 2 .===========
\begin{enumerate}

\begin{flushleft}
{\large 2. The product of two odd integers is odd.}
\end{flushleft}

\begin{itemize}

\item \textbf{Solution}\\
\large 1. odd number = 2n+1\\
\large 2. product of two odd numbers $= (2n+1)(2n+1)$\\
\large 3. $= 4n^2 +4n+1$\\
\large 4. $= 2(2n^2+2n)+1$\\
\large 5. Dividing $ [2(2n^2+2n)+1]$ by 2 the remainder will always be 1\\
\large 6. \therefore \hspace{.2cm} $Odd$\\

\end {itemize}
\end {enumerate}

%========= Problem 2 .===========
\begin{enumerate}

\begin{flushleft}
{\large 3. If $n^3 +5$ is odd then n is even, for any n \in \mathbb{Z} }
\end{flushleft}

\begin{itemize}

\item \textbf{Solution}\\
\large 1. lets say P: n is an integer and ($n^3+5$) is odd\\
\large 2. \neg $P: n is an integer and ($2^3+5$) is even$\\
\large 3. $lets say q: n is even$\\
\large 4. \neg $q: n is odd$\\
\large 5. $By contraposition: if n is odd, then n is an integer and (n^3+5) is even.$\\

\end {itemize}
\end {enumerate}


%========= Problem 3 .===========
\begin{enumerate}

\begin{flushleft}
{\large 4. If $3n+2$ is even then n is even, for any n \in \mathbb{Z} }
\end{flushleft}

\begin{itemize}

\item \textbf{Solution}\\
\large 1. Suppose that 3n+2 is even and n is odd\\
\large 2. recall n is odd, and then product of two odd number is odd\\
\large 3. this means 3n is odd and that 3n+2 is odd\\
\large 4. By contradiction this is proven that both are even\\


\end {itemize}
\end {enumerate}


%========= Problem 4 .===========
\begin{enumerate}

\begin{flushleft}
{\large 5. The sum of a rational number and an irrational number is irrational. }
\end{flushleft}

\begin{itemize}

\item \textbf{Solution}\\
\large 1. suppose that m is rational and n is irrational\\
\large 2. it is enough to prove that number k=m+n is an irrational\\
\large 3. we conversely assume that the sum of k=m+n is rational\\
\large 4. so clearly if m is rational then -m is rational\\
\large 5. since the sum of the rational numbers are rational we consider\\
k+(-m)=m+n+(-m) [since -m is rational]\\
\large 6. = \cancel {m}+n-\cancel{m}\\
\large 7. = n\\
\large 8. \therefore \hspace{.2cm} $k+(-m) = n, is rational but by supposition the number n is irrational$\\


\end {itemize}
\end {enumerate}




%-----------        ----------- 
%----------- Part 2 ----------- 
%-----------        ----------- 


\begin{flushleft}
{\Large 1. Mathematical Proofs}
\end{flushleft}



%========= Problem 1 .===========
\begin{enumerate}

\begin{flushleft}
{\large 1.How many different arrangements of the English alphabet are there?}
\end{flushleft}

\begin{itemize}

\item \textbf{Solution}\\
\large 1. There are 26 letters in the English alphabet\\
\large 2. With different arrangements means permutation of 26 letters\\
\large all at a time = 26!


\end {itemize}
\end {enumerate}


%========= Problem 2 .===========
\begin{enumerate}

\begin{flushleft}
{\large 2. There are 18 mathematics majors and 325 computer science majors at a college. In how many ways can two representatives be picked so that one is a mathematics major and the other is a computer science major?}
\end{flushleft}

\begin{itemize}

\item \textbf{Solution}\\
\large 1. Using product rule 18*325 = 5850\\


\end {itemize}
\end {enumerate}


%========= Problem 3 .===========
\begin{enumerate}

\begin{flushleft}
{\large 3. A particular brand of shirt comes in 12 colors, has a male version and a female version, and comes in
three sizes for each sex. How many different types of this shirt are made?}
\end{flushleft}

\begin{itemize}

\item \textbf{Solution}\\
\large 1. We have 2 types of shirt (M/F)\\
\large 2. We have 3 different sizes(S/M/L)\\
\large 3. We have 12 different types of shirts\\
\large 4. using product rule we get 12 x 3 x 2 = 72\\


\end {itemize}
\end {enumerate}

\pagebreak 
%========= Problem 4 .===========
\begin{enumerate}

\begin{flushleft}
{\large 4. A multiple-choice test contains 10 questions. There are four possible answers for each question. In how many ways can a student answer the questions on the test if the student answers every question?}
\end{flushleft}

\begin{itemize}

\item \textbf{Solution}\\
\large 1. Assuming there is one correct answer \\
\large 2. Each questions can be answered in 4 possible ways \\
\large 3. So if the student has 10 questions with 4 options each\\
\large 4. Which gives us $4^{10}$ variations\\
\large 5. \therefore \hspace{.2cm}$1,048,576 different variations. $\\


\end {itemize}
\end {enumerate}


%========= Problem 5 .===========
\begin{enumerate}

\begin{flushleft}
{\large 5. Suppose we have the same multiple choice test as described in question 4, but we relax the assumption that the student has to answer all questions. In other words, how many ways are there for a student answer the questions on the test if the student can leave answers blank?}
\end{flushleft}

\begin{itemize}

\item \textbf{Solution}\\
\large 1. If the students leave something blank than that means they have 5 options per questions\\
\large 2. this gives us $5^{10}$ options\\
\
\large 3. \therefore \hspace{.2cm}$9,765,625 variations$\\


\end {itemize}
\end {enumerate}




%-----------        ----------- 
%----------- Part 3 ----------- 
%-----------        ----------- 


\begin{flushleft}
{\Large 1. Set Theory}
\end{flushleft}



%========= Problem 1 .===========
\begin{enumerate}

\begin{flushleft}
{\large 1. Are these pairs of sets equal or not: \{5, 1, 3\}, \{1, 3, 5\}. You just have to answer Yes or No.}
\end{flushleft}

\begin{itemize}

\item \textbf{Solution}\\
\large • YES

\end {itemize}
\end {enumerate}

%========= Problem 2 .===========
\begin{enumerate}

\begin{flushleft}
{\large 2. What is the cardinality of (the number of elements in) the set {a, {a}, {a, {a}}}.}
\end{flushleft}

\begin{itemize}

\item \textbf{Solution}\\
\large • Cardinality is 3

\end {itemize}
\end {enumerate}

%========= Problem 3 .===========
\begin{enumerate}

\begin{flushleft}
{\large 3. Find the power set of \{a, \{a, b\}\}, where a and b are distinct elements. Your answers must be sets.
That is, they should start with \{ and end with \}}
\end{flushleft}

\begin{itemize}

\item \textbf{Solution}\\
\large • \{\emptyset,\{a\},\{a,b\},\{a,\{a,b\}\}\}

\end {itemize}
\end {enumerate}

%========= Problem 4.===========
\begin{enumerate}

\begin{flushleft}
{\large 4. Let A = {a, b, c} and B = {y, z}. Determine the following set. Your answers must be sets. That is,
they should start with \{ and end with \} (0.2 points each). \}}
\end{flushleft}



 \emph{A) AxB}\\
 \begin{itemize}
\item \textbf{Solution}\\
\large \{(a, y), (a, z), (b, y), (b, z), (c, y), (c, z)\}\\
\end {itemize}

 \emph{B) BxA}\\
 \begin{itemize}
\item \textbf{Solution}\\
\large \{(y, a), (y, b), (y, c), (z, a), (z, b), (z, c)\}\\


\end {itemize}
\end {enumerate}


%========= Problem 5.===========
\begin{enumerate}

\begin{flushleft}
{\large 5. Let A = \{1, 2, 3, 4, 5\} and B = \{0, 3, 6\}. Determine the following sets. Note that your answers must
be sets. That is, they should start with \{ and end with \}}
\end{flushleft}



 \emph{$A) A \cup B$}\\
 \begin{itemize}
\item \textbf{Solution}\\
\large \{0,1,2,3,4,5,6\}\\
\end {itemize}


 \emph{$B) A \cap B$}\\
 \begin{itemize}
\item \textbf{Solution}\\
\large \{3\}\\

\end {itemize}

 \emph{$C) B - A $}\\
 \begin{itemize}
\item \textbf{Solution}\\
\large \{0,6\}\\

\end {itemize}
\end {enumerate}



%========= Problem 6.===========
\begin{enumerate}

\begin{flushleft}
{\large 6. Let A = \{0, 2, 4, 6, 8, 10\}, B = \{0, 1, 2, 3, 4, 5, 6\}, and C = \{4, 5, 6, 7, 8, 9, 10\}. Determine the following sets. Your answers must be sets. That is, they should start with \{ and end with \}}
\end{flushleft}



 \emph{$A) A \cap B \cap C$}\\
 \begin{itemize}
\item \textbf{Solution}\\
\large \{4,6\}\\
\end {itemize}

 \emph{$B) (A \cup B)\cap C$}\\
 \begin{itemize}
\item \textbf{Solution}\\
\large \{0,1,2,3,4,5,6,7,8,9,10\}\\



\end {itemize}
\end {enumerate}

%========= Problem 7.===========
\begin{enumerate}

\begin{flushleft}
{\large 7.Determine the set  $\bigcup\limits_{i=1}^{\infty} A_{i}$ if, for every positive integer \iota }
\end{flushleft}



 \emph{$A) A_{i} = \{ - \iota, - \iota +1, ..., -1,0,1,...,\iota -1, \iota\}$}\\
 \begin{itemize}
\item \textbf{Solution}\\
\large $\bigcup\limits_{i=1}^{\infty} A_{i} = \{..., -3,-2,-1,0,1,2,3,...\}$\\
\end {itemize}


 \emph{$B) A_{i} = \{ - \iota, \iota\}$}\\
 \begin{itemize}
\item \textbf{Solution}\\
\large $\bigcup\limits_{i=1}^{\infty} A_{i} = \{..., -4,-3,-2,-1,1,2,3,4,...\}$\\


\end {itemize}
\end {enumerate}

%========= Problem 7.===========
\begin{enumerate}

\begin{flushleft}
{\large 8. Determine the set  $\bigcap\limits_{i=1}^{\infty} A_{i}$ if, for every positive integer \iota }
\end{flushleft}

 \emph{$A) A_{i} = \{ - \iota, - \iota +1, ..., -1,0,1,...,\iota -1, \iota\}$}\\
 \begin{itemize}
\item \textbf{Solution}\\
\large $\{-1,0,1\} $\\
\end {itemize}


 \emph{$B) A_{i} = \{ - \iota, \iota\}$}\\
 \begin{itemize}
\item \textbf{Solution}\\
\large $ \{-1,1 \} \cap \{-2,2\} \cap \{-3,3\} \cap ...\{-\iota, \iota\} \cap ...\{ - \infty, \infty\} $\\
\large • or simply \emptyset\\


\end {itemize}
\end {enumerate}



\end{document} 