\documentclass[11pt]{article}
\documentclass[11pt]{article}
\documentclass{article}
\usepackage{libertine}
\usepackage{amssymb}
\usepackage{amsmath}
\usepackage[makeroom]{cancel}
\usepackage{indentfirst}

% To produce a letter size output. Otherwise will be A4 size.
\usepackage[letterpaper]{geometry}
% To produce a letter size output. Otherwise will be A4 size.
\usepackage[letterpaper]{geometry}

% For enumerated lists using letters: a. b. etc.
\usepackage{enumitem}

\topmargin -.5in
\textheight 9in
\oddsidemargin -.25in
\evensidemargin -.25in
\textwidth 7in
\usepackage{pifont}

\begin{document}





\begin{center}
    {\Large Valentinno Cruz\\
    Homework Assignment 5\\
    April. 21st, 2021\\}

\end{center}

%-----------             ----------- 
%----------- ALGORITHMS  ----------- 
%-----------             ----------- 

\begin{flushleft}
{\Large 1. Algorithms}
\end{flushleft}



%========= Problem 1 .===========


\begin{enumerate}

\begin{flushleft}
{\large 1. Develop an algorithm that takes as input a list of n integers and finds the location of the last odd
integer in the list or returns −1 if there are no odd integers in the list.}\\
\end{flushleft}





\begin{itemize}

\item \textbf{Solution}\\
\large 1. Declare int x as variable \\
\large 2. Get input for x\\
\large 3. algorithm takes list and size x\\
\large 4. Algorithm input(list,x)\\
\large 5. set index to -1 \\
\large 6. Declare an array of x ints\\
\large 7. loop through list\\
\large 8. for i from 0 to n-1\\
\large \hspace*{10mm} check if number in index i is odd. yes\\
\large \hspace*{5mm} end for loop \\
\large 9. for i = 1 to x\\
\large \hspace*{10mm} if list[i] \%2 == 1 \\
\large \hspace*{15mm} set index = i\\
\large \hspace*{10mm} end if \\
\large \hspace*{5mm}end for\\
\large 10. return index\\


\end {itemize}
\end {enumerate}


%========= Problem 2 .===========


\begin{enumerate}

\begin{flushleft}
{\large 2. Develop an algorithm that inserts an integer a in the appropriate position into the list $x_1, x_2, ..., x_n$ of
integers that are in decreasing order.}\\
\end{flushleft}





\begin{itemize}

\item \textbf{Solution}\\
\large 1. declare a list an array of x ints\\
\large 2. for i = 1 to x\\
\large \hspace*{10mm} get the input of the ith index of the list\\
\large \hspace*{5mm} end for\\
\large 4. declare an int z and get int of z\\
\large 5. set a flag to false\\
\large \hspace*{10mm} if z $\geq$ list[i] \\
\large \hspace*{15mm} insert z at the position i on the list\\
\large \hspace*{15mm} assign flag to true \\
\large \hspace*{15mm} otherwise break loops\\
\large \hspace*{10mm} end if \\
\large \hspace*{5mm} end for\\ 
\large 6. if the flag is false \\
\large \hspace*{10mm} put z at last position of the list \\
\large \hspace*{5mm} end if\\

\end {itemize}
\end {enumerate}


\pagebreak




%-----------                     ----------- 
%----------- Order of Complexity ----------- 
%-----------                     ----------- 

\begin{flushleft}
{\Large 2. Order of Complexity}
\end{flushleft}



%========= Problem 1 .===========


\begin{enumerate}

\begin{flushleft}
{\large 1. State whether each of these functions is or is not $O(x^2)$. If you state that the function is not $O(x^2)$
then that is all you need to do. If you state that the function is $O(x^2)$ then find witnesses $C$ and $k$
such that $f(x) \leq Cg(x)$ where $g(x) = x^2$ when $x > k.$}\\
\end{flushleft}


%============ part A ============ 
\large (a) $f(x) = 17x + 11$\\


\begin{itemize}

\item \textbf{Solution}\\
\large  $f(x) = 17x+11$ which is $O(x)$ this gives us $O(n^2)$\\
\large  this means it is $O(n^2)$\\
\large which means we get $C$=28 and $K$ =1\\
\large \therefore $ 17x+11 \leq 28*x^2$ for all $x > 1$\\


\end {itemize}

%================================
%============ part B ============ 
%================================

\large (b) $f(x) = xlog_2 x$\\


\begin{itemize}

\item \textbf{Solution}\\
\large  We have $O(xlnx)$\\
\large this means we have $O(x^2)$\\
\large we can assign $C$ and $K$ to 1\\
\large \therefore $ xlnx \leq 1*x^2$ for every $x \geq 1$\\


\end {itemize}


%================================
%============ part C ============ 
%================================

\large (c) $f(x) = x^4/2$\\


\begin{itemize}

\item \textbf{Solution}\\
\large  This is $O(x^4)$ not $O(x^2)$\\
\large $\therefore $ not possible\\



\end {itemize}
\end {enumerate}

%================================
%========= BIG-O .===============
%================================


\begin{enumerate}

\begin{flushleft}
{\large 2. Give a big-$O$ estimate (in terms of $n$) for the number of additions used in this segment of an algorithm}\\
\end{flushleft}

\begin{itemize}

\item \textbf{Solution}\\
\large the value of $i$ will iterate from $1$ till $n = O(n)$\\
\large so for each value of $i$ the $j$ value will iterate from $0$ to $n = O(n)$\\
\large $\therefore$ the complexity is $O(n)*O(n)$\\
\large which gives us $O(n^2)$\\


\end {itemize}



\end {enumerate}




\pagebreak 

%========= Problem 3 .===========


\begin{enumerate}

\begin{flushleft}
{\large 3. Suppose you have a computer which takes $10^−9$ seconds to perform a bit operation. What is the
largest problem, in terms of $n$, that this computer can solve in one second using an algorithm that
requires $f(n)$ bit operations if}\\
\end{flushleft}


%============ Problem A ============ 
\large (a) $f(n) = log_2 n$\\


\begin{itemize}

\item \textbf{Solution}\\
\large  $f(n) = log_2 n$ \\
\large this means $log_2 n \leq 10^9$\\
\large so $2^{log_2n}$ $\leq$ $2^{10^9}$\\
\large $\therefore n = 2^{10^9}$\\
\end {itemize}


%============ Problem B ============ 
\large (b) $f(n) = n$\\


\begin{itemize}

\item \textbf{Solution}\\
\large  $f(n) = log_2 n$ \\
\large this means $n \leq 10^9$\\


\end {itemize}


%============ Problem C ============ 
\large (c) $f(n) = 2^n$\\


\begin{itemize}

\item \textbf{Solution}\\
\large  $f(n) = 2^n$\\
\large which means $2^n \leq 10^9$\\
\large so $log 2^n \leq log 10^9$\\
\large $nlog2 \leq log10^9$\\
\large $n \leq 9log 10/log 2$ \\
\large which is $9/log2 \simeq 30$\\
\large $\therefore n < 30$ \\


\end {itemize}


%============ Problem D ============ 
\large (d) $f(n) = n!$\\


\begin{itemize}

\item \textbf{Solution}\\
\large  $n! \leq 10^9$\\
\large we know that $11! = 39,916,800$ which is $\leq 10^9$\\
\large so $12! = 479,001,300$ is $\leq 10^9$\\
\large and $13! = 627,020,800$ which is $\geq 10^9$\\
\large $\therefore n = 12$\\


\end {itemize}








\end {enumerate}
\end{document} 